\documentclass[12pt, letterpaper, titlepage]{article}
\usepackage[left=3.5cm, right=2.5cm, top=2.5cm, bottom=2.5cm]{geometry}
\usepackage[MeX]{polski}
\usepackage[utf8]{inputenc}
\usepackage{graphicx}
\usepackage{enumerate}
\usepackage{amsmath} %pakiet matematyczny
\usepackage{amssymb} %pakiet dodatkowych symboli
\title{porcjowanie składników}
\author{Damian Kowalski}
\begin {document}
\maketitle

\newpage

\begin{table}[h]
\centering\caption{Bramka logiczna NOT}

\begin {tabular}{|c|c|c|}
\hline 
we & wyj\\
\hline
0 & 1\\
\hline
1 & 0\\
\hline
\end{tabular}
\end{table}


\begin{table}[h]
\centering\caption{Bramka logiczna AND}
\begin {tabular}{|c|c|c|}
\hline
A & B & Wyj\\
\hline
0 & 1 & 0\\
\hline
0 & 0 & 0\\
\hline
1 & 1 & 1\\
\hline
1 & 0 & 0\\
\hline
\end{tabular}
\end{table}

\begin{table}[h]
\centering\caption{Bramka logiczna NAND}
\begin {tabular}{|c|c|c|}
\hline
A & B & Wyj\\
\hline
0 & 1 & 1\\
\hline
0 & 0 & 1\\
\hline
1 & 1 & 0\\
\hline
1 & 0 & 1\\
\hline
\end{tabular}
\end{table}

\begin{table}[h]
\centering\caption{Bramka logiczna OR}
\begin {tabular}{|c|c|c|}
\hline
A & B & Wyj\\
\hline
0 & 0 & 0\\
\hline
0 & 1 &	1\\
\hline
1 & 1 & 1\\
\hline
1 & 0 & 0\\
\hline
\end{tabular}
\end{table}

\begin{table}[h]
\centering\caption{Bramka logiczna NOR}
\begin {tabular}{|c|c|c|}
\hline
A & B & Wyj\\
\hline
0 & 1 & 1\\
\hline
0 & 0 & 1\\
\hline
1 & 1 & 0\\
\hline
1 & 0 & 1\\
\hline
\end{tabular}
\end{table}

\end{document}