\documentclass[12pt, letterpaper, titlepage]{article}
\usepackage[left=3.5cm, right=2.5cm, top=2.5cm, bottom=2.5cm]{geometry}
\usepackage[MeX]{polski}
\usepackage[utf8]{inputenc}
\usepackage{graphicx}
\usepackage{enumerate}
\usepackage{amsmath} %pakiet matematyczny
\usepackage{amssymb} %pakiet dodatkowych symboli
\title{porcjowanie składników}
\author{Damian Kowalski}
\begin {document}
\maketitle

\newpage
\centering
Ułamek w tekście
$$ \frac{1} {x} $$\\
Równanie
$$ c^{2}=a^{2}+b^{2} $$

\begin{equation}
\frac{1}{x}
\end{equation}
\\
\begin{equation}
c^{2}=a^{2}+b^{2}
\end{equation}
\\
Indeks górny
$$ x^{y} \ e^{x} \ 2^{e} \ A^{2 \times 2} $$\\
Indek dolny
$$ x_{y} \ e_{x} \ 2_{e} \ A_{2 \times 4} $$\\
Oba Indeksy
$$ x_{y}^{y} \ 2_{e}^{e} \ A_{2 \times 3}^{2 \times 3} $$\\

$$ \sqrt{ \frac{2^{n}}{3_{n}}} \neq \sqrt[\frac{1}{3}]{1+n} $$\\

$$ \sum \ \sum_{i=1}^{10}x_i \ \int \bigcap \ \bigcup \ \bigsqcup \ \bigvee \ \bigwedge $$
\end{document}