\documentclass[12pt, letterpaper, titlepage]{article}
\usepackage[left=3.5cm, right=2.5cm, top=2.5cm, bottom=2.5cm]{geometry}
\usepackage[MeX]{polski}
\usepackage[utf8]{inputenc}
\usepackage{graphicx}
\usepackage{enumerate}
\usepackage{amsmath} %pakiet matematyczny
\usepackage{amssymb} %pakiet dodatkowych symboli
\title{Domowe przepisy}
\author{Damian Kowalski}
\date{Październik 2022}
\begin{document}
\maketitle

\newpage
\fontsize{25}{5}{\selectfont
\underline{Przepis na gołąbki}}
\par
\begin{enumerate}
\item 700 g mielonej łopatki wieprzowej
\item główka kapusty - białej lub włoskiej
\item 1,5 szklanki bulionu lub odlanego rosołu
\item 1 średnia cebula - około 200 g
\item łyżka oleju roślinnego
\item 2 łyżki bułki tartej
\item 2 łyżki przecieru pomidorowego lub koncentratu
\item przyprawy: po płaskiej łyżeczce soli i pieprzu
\end{enumerate}

\newpage
{\fontsize{20}{5}{\selectfont
\textbf{Gołąbki - moj sposob przygotowania} }}
\\ \\ Szklanka ma u mnie pojemność 250 ml. 
Warzywa ważone były przed ewentualnym obraniem/przygotowaniem.
\\
Kalorie policzone zostały na podstawie użytych przeze mnie składników. Jest to więc orientacyjna ilość kalorii liczona na podstawie produktów, których użyłam w danym przepisie. Z podanej ilości składników otrzymasz około 17 gołąbków. 
\\
Przed szykowaniem gołąbków zachęcam też do przeczytania najpierw całego wpisu. W treści podaję bowiem dużo ciekawych porad dotyczących składników oraz ich zamienników. Być może zapragniesz użyć np. kapusty pekińskiej zamiast zwykłej białej lub dodać do farszu ryż. Na koniec opisuję również jak zrobić gołąbki z piekarnika.
\\
\\
\\
{\fontsize{20}{5}{\selectfont
\underline{Porady} }}
\\
\begin{enumerate}
\item Gołąbki umieść na środkowej półce w piekarniku nagrzanym do 180 stopni
\item Ustaw opcję pieczenia góra/dół
\item Naczynie z gołąbkami przykryj i piecz przez 50 minut
\item Po tym czasie zalej je sosem pomidorowym
\item Sos pomidorowy to np: 1 szklanka bulionu; 2 szklanki przecieru pomidorowego; 4 łyżki koncentratu; po łyżeczce soli i oregano oraz pół łyżeczki pieprzu
\item 2 łyżki bułki tartej
\item Gołąbki z sosem umieść ponownie w piekarniku i zapiekaj je już bez przykrycia przez około 20 minut lub dłużej
\end{enumerate}

\end{document}
