\documentclass[12pt, letterpaper, titlepage]{article}
\usepackage[left=3.5cm, right=2.5cm, top=2.5cm, bottom=2.5cm]{geometry}
\usepackage[MeX]{polski}
\usepackage[utf8]{inputenc}
\usepackage{graphicx}
\usepackage{enumerate}
\usepackage{amsmath} %pakiet matematyczny
\usepackage{amssymb} %pakiet dodatkowych symboli
\title{porcjowanie składników}
\author{Damian Kowalski}
\begin {document}
\maketitle

\newpage

\begin{table}[h]
\centering\caption{Przykładowy system decyzyjny (U,A,d), modelujący problem diagnozy medycznej, której efektem jest decyzja lub nie wykonaniu operacji wycięcia wyrostka robaczkowego,
U=\{u_{1},u_{2},...,u_{10}\}, A=\{a_{1},a_{2}\},d\in D=\{TAK, NIE\}}

\begin {tabular}{c|ccc}
	\hline
	\hline
	Pacjent & Ból brzucha & Temperatura ciała & Operacja\\
	\hline
	u1 & Mocny & Wysoka & Tak\\
	u2 & Średni & Wysoka & Tak\\
	u3 & Mocny & Średnia & Tak\\
	u4 & Mocny & Niska & Tak\\
	u5 & Średni & Średnia & Tak\\
	u6 & Średni & Średnia & Nie\\
	u7 & Mały & Wysoka & Nie\\
	u8 & Mały & Niska & Nie\\
	u9 & Mocny & Niska & Nie\\
	u10 & Mały & Średnia & Nie\\
	\hline
	\hline
\end{tabular}
\end{table}
\end{document}